\documentclass[12pt]{article}
%\documentclass{book}
\usepackage{pstricks}
\usepackage{pst-plot}
\usepackage{pst-node}
\usepackage{pst-tree}
\usepackage{pst-coil}
\usepackage[absolute]{textpos}
\usepackage{color}
\usepackage{amssymb}
\usepackage{graphicx}
\usepackage{hyperref}
\usepackage{xspace}
\usepackage{multirow}
\usepackage{lscape}
\usepackage{fancyhdr}
\usepackage{lastpage}
\usepackage{longtable}
\usepackage{arydshln}
\usepackage{xcolor}
\usepackage[tikz]{bclogo}
\usepackage[framemethod=tikz]{mdframed}
\usepackage{lipsum}
\usepackage[many]{tcolorbox}
\usepackage{geometry}

\definecolor{bgblue}{RGB}{245,243,253}
\definecolor{ttblue}{RGB}{91,194,224}

\begin{document}
One of the most confusing aspects of Lagrangian/Hamiltonian mechanics centers on the energy $E$, the function $h$ (or the Hamiltonian $H$), to what extent these two terms can be identified with each other, and which of them is conserved.  The opportunity for confusion arises because there are many different possibilities that show up in practice.  The diagram below attempts to list, organize, and group the most common possibilities, but since the Lagrangian method is very flexible, no doubt, cases exist which fall outside the .



To my knowledge, no textbook or article lays out all five cases in a clear fashion, comparing and contrasting them along the way.  The likely reason for this 'deficiency' is that most of the physics literature focuses on systems of a traditional nature in which the Hamiltonian is not time-dependent.  There are excellent reasons for this at the fundamental level, since the basic four forces seem to be time-invariant.  But to the student grappling with celestial mechanics (a subject that has just about completely fallen out of favor in physics departements) it is of crucial importance; the circular restricted three-body problem being the textbook example.  

The aim of this column is to address this gap.  It is important to acknowlege that <David Morin's treatment of this topic> (http://www.people.fas.harvard.edu/~djmorin/chap15.pdf) is an excellent resource and my treatment largely follows his, with some new extensions and clarifications.  In addition, some examples and concepts (buried in homework problems strewn throughout various chapters) are mined from Herbert Goldstein's textbook 'Classical Mechanics', 2nd edition.

The single most important 
\[ E = T + V = \frac{1}{2} m \dot {\vec v} ^2 + V(\vec r, \vec v; t) \]

\[ p_i = \frac{\partial L}{\partial {\dot q}^i} \]

\[ h = \frac{\partial L}{\partial {\dot q}^i} {\dot q}^i - L(q,\dot q;t) \]

\[ \frac{d h}{dt} = \left( \frac{d}{dt} \frac{\partial L}{\partial {\dot q}^i} \]

The Euler Lagrange equations are invariant to a basic change in coordinates from $q^i$ to
$y^j$ of the form 
\begin{equation}\label{ELinv_b}
  q^i = q^i(y^j) .
\end{equation}
To see this first note that the from the 
form of the transformation equation (\ref{ELinv_b}) we get
\begin{equation}\label{ELinv_a}
  {\dot q}^i = \frac{\partial q^i}{\partial y^j} {\dot y}^j \,
\end{equation}
where $\dot f = \frac{d}{dt} f$. Next note that the Lagrangian 
$\tilde L (y^j,{\dot y}^j;t )$ in the $y^j$ coordinates 
is related to the Lagrangian $L (q^i,{\dot q}^i;t )$ 
in $q^i$ coordinates by virtue of a substitution of (\ref{ELinv_b}) 
and (\ref{ELinv_a}) yielding
\begin{equation}\label{ELinv_c}
  \tilde L (y^j,{\dot y}^j;t ) = L (q^i(y^j),{\dot q}^i(y^j,{\dot y}^j);t ) .
\end{equation}
The parts of the Euler-Lagrange equation in terms
of the $y^j$ coordinates in relation to the $q^i$ coordinates are
\begin{equation}\label{ELinv_d}
  \frac{\partial \tilde L}{\partial y^j} =    \frac{\partial L}{\partial q^i}        \frac{\partial q^i}{\partial y^j} 
                                        +  \frac{\partial L}{\partial {\dot q}^i} \frac{\partial {\dot q}^i}{\partial y^j}
\end{equation}
and
\begin{equation}\label{ELinv_e}
  \frac{\partial \tilde L}{\partial {\dot y}^j} = \frac{\partial L}{\partial {\dot q}^i} \frac{\partial {\dot q}^i}{\partial {\dot y}^j} .
\end{equation}
Now substituting (\ref{ELinv_d}) and (\ref{ELinv_e}) into the Euler-\\Lagrange equations
yields
\begin{eqnarray}\label{ELinv_f}
  \frac{d}{dt} \left( \frac{\partial \tilde L}{\partial {\dot y}^j } \right) - \frac{\partial \tilde L}{\partial y^j} 
    & = &   \frac{d}{d t} \left( \frac{\partial L}{\partial {\dot q}^i} \right) \frac{\partial {\dot q}^i}{\partial {\dot y}^j} \\ \nonumber
	&   & + \frac{\partial L}{\partial {\dot q}^i} \frac{d}{dt} \left( \frac{\partial {\dot q}^i}{\partial {\dot y}^j} \right) \\ \nonumber
    & = & - \frac{\partial L}{\partial q^i}        \frac{\partial q^i}{\partial y^j} 
	      - \frac{\partial L}{\partial {\dot q}^i} \frac{\partial {\dot q}^i}{\partial y^j} .\nonumber
\end{eqnarray}
But from (\ref{ELinv_a}) $ {\partial {\dot q}^i}/{\partial {\dot y}^j} = {\partial q^i}/{\partial y^j}$ and thus the second and
fourth terms in (\ref{ELinv_f}) cancel, leaving
\begin{equation}\label{ELinv_g}
  \frac{d}{dt} \left( \frac{\partial \tilde L}{\partial {\dot y}^j } \right) - \frac{\partial \tilde L}{\partial y^j} = 
  \left[ \frac{d}{d t} \left( \frac{\partial L}{\partial {\dot q}^i} \right) - \frac{\partial L}{\partial q^i} \right] 
  \frac{\partial q^i}{\partial y^j} .
\end{equation}
Using the definition $\Xi_i = \frac{d}{dt}\left(\frac{\partial L}{\partial \dot q^i} \right) - \frac{\partial L}{\partial q^i}$,
Eq. (\ref{ELinv_g}) takes on the more obvious form of
\begin{equation}\label{ELinv_h}
  \Xi_{\tilde j} = \Xi_i {\Lambda^i}_{\tilde j}
\end{equation}
which shows that the Euler-Lagrange equations transform like the components of a covariant vector.


\end{document}
