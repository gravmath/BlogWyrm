\documentclass[12pt]{article}
\usepackage{amssymb}
\usepackage{graphicx}
\usepackage{hyperref}
\usepackage{xspace}
\usepackage{multirow}


\begin{document}
The concept of the Turing Test as the basic hurdle that an artificially intelligent system has to overcome in order to be judged sufficiently like a human is both pervasive and intriguing.

Usually, these hurdles are viewed as question of evolution - of smartening the AI so that it acts like a human being.  Topics along this line include enabling an AI so that levels of evocation can be recognized; an essential property that allows for understanding humor and getting double entrendres.  Poetry and evocative imagery is also a complication that has been explored off and on in both serious academic circles and fanciful science fiction avenues.   

The Birthday Puzzle

Often called the Birthday Paradox, this puzzle is a significant challenge to the basic intuition that each of us has.  As described, the Birthday Puzzle, goes something like this.  Suppose that there are $n$ persons in a room, say attending a party.  What is the probability that any two of them have the same birthday?

The human reaction is that you need a large number of people 


\end{document}
